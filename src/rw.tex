\section{Related Work}

\lightlipsum[1]

\subsection{Predictive motion models}

\begin{itemize}
	\item TODO : Discuss kinematic, quasi-dynamic, dynamic and data-driven
\end{itemize}

% Kinematic
The most commonly used motion models for \acp{UGV} are purely kinematic models.
Such models have the advantage that they are computationally simple and are relatively straightforward to train.
Since ideal models make the assumption that the \ac{UGV} operates in pure rolling and no slippage conditions, such models tend to have high motion prediction error.
For the case of \acp{SSMR}, which inherently require high wheel slippage and skidding to steer, \citet{Mandow2007} proposed the the \ICRBASED model, an extension of the \ac{IDD} model.
This model assumes that each robot wheel has a distinct \ac{ICR} due to slippage and that these remain constant for a given terrain. 
The authors have driven human-generated trajectories of about \SI{30}{\meter} and used \ac{RTK}-\ac{GNSS} to generate ground-truth to empirically fit model parameters.
\citet{Wang2015} later extended the \ICRBASED model, suggesting that wheel \ac{ICR} values based on the \ac{SSMR}'s \ac{ROC}. 
Alternatively, \citet{Anousaki2004} proposed to use empirical data to tune all coefficients of the Jacobian matrix, allowing to take wheel-terrain interactions into the motion model.
The original goal of this work is to provide a dead-reckoning motion prediction solution for \ac{SLAM} frameworks. 
In this work, the model was validated on a \ac{SSMR} on a human-driven trajectory of under \SI{50}{\meter}.
More recently, we have proposed an experimental evaluation of kinematic motion models for \acp{SSMR} operating on asphalt and snow terrains~\citep{Baril2020}. 
In this work, we show that kinematic models should be calibrated based on the desired prediction window and operating terrain. 
While kinematic models offer simplicity, they are often limited in prediction accuracy~\citep{Seegmiller2016}. 
% TODO: Find simple kinematic model for Ackerman vehicles?

% Dynamic
Pushing modeling richness further, full dynamic models have been explored to allow to model traction force between \ac{UGV} wheels and soil. 
\citet{Yu2009} have proposed a dynamic model for a \ac{SSMR} in general planar (2D) and a distinct model for linear 3D motion. 
This model allows to account for rolling resistance, coefficient of friction, shear deformation modulus and gravity when predicting \ac{UGV} motion.
To train and evaluate this model, the authors have commanded a \ac{SSMR} to conduct constant circular motions.
\citet{Seegmiller2016} later proposed a unified notation allowing to represent and model \acp{UGV} as multi-body dynamic systems.
Such models allow the user to specify suspension, actuation and wheel traction models that govern the model's dynamic equations. 
Their model evaluation is conducted on human-generated trajectories on asphalt, dirt and grass, using \ac{RTK}-\ac{GNSS} and an \ac{IMU} to generate the ground-truth.
\citet{Yang2022} later extended this model formulation to allow modeling single-input-multi-output joints, adapting the model to a Mars rover.
In this work, a digital elevation map is used with terrain properties is used to increase motion prediction accuracy.
These models have been trained and evaluated for multiple human manoeuvres on a simulated terrain, forcing high traction variations. 
While proving to be more accurate, dynamic models are computationally complex, making it difficult to predict vehicle motion fast enough for modern \ac{MPC} approaches~\citep{Williams2018}. % TODO: find a better citation for this
Additionally, such models require extensive information on the platform used~\citet{Yang2022}, which is typically given for extra planetary rovers but hard to find for commercial platforms.

% Quasi-dynamic
Aiming to find an optimal trade-off between full kinematic and dynamic models, multiple approaches have been proposed to allow dynamics-aware modeling without requiring to solve complete differential equations.
We will refer to such models as Quasi-dynamic.
\citet{Seegmiller2014} have proposed to account for wheel slip by subtracting slip velocity to the predicted body velocity of a slip-less model, such as the \ac{IDD} or bicycle model.
In this model, empirically-tuned linear coefficients allow to compute slip velocity from slip-less velocity based.
These coefficients are based on physical quantities that the authors have found to affect slip velocities.
The authors show that under near steady-state conditions of normal operation, this model is nearly as accurate as a full dynamic model, while requiring significantly less computation time and physical information on the given platform~\citep{Seegmiller2014}.
The data used to train and validate this model is based on the same human-generated trajectories as in~\citep{Seegmiller2016}.
Similarly,~\citep{Ostafew2016} suggested to model the slip velocity as a \ac{GP}.


Quasi-dynamic models
\citep{Seegmiller2014}
\citep{Ostafew2016}
\citep{Rabiee2019}
\citep{Takemura2021}
Mention \citep{Seegmiller2013} for model identification.

Data-driven models
\citep{Williams2018}
\citep{Nagariya2020}
\citep{Tremblay2021}

\lightlipsum[1-3]

Datasets
\citep{Triest2022}


\subsection{Model training}

\begin{itemize}
	\item TODO : Discuss IPEM, Kalman Filters, datasets (Tartan, JFT, etc.) our method is applicable to any vehicle as long as...
\end{itemize}

\lightlipsum[1-3]
