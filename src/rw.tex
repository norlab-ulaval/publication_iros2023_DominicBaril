\section{Related Work}

\lightlipsum[1]

\subsection{Predictive motion models}

\begin{itemize}
	\item TODO : Discuss kinematic, quasi-dynamic, dynamic and data-driven
\end{itemize}

The computationally most simple models that have been proposed fall into the kinematic category.
Such models make use of a Jacobian matrix to compute the resulting robot body-frame velocities resulting from inputs.
\citet{Anousaki2004} were the first to suggest to use empirical data to tune all coefficients of the Jacobian matrix, allowing to take wheel-terrain interactions into the motion model.
The original goal of this work is to provide a dead-reckoning motion prediction solution for \ac{SLAM} frameworks. 
In this work, the model was validated on a \ac{SSMR} on a human-driven trajectory of under \SI{50}{\meter}.
\citet{Mandow2007} later proposed the \ICRBASED model, however this time by modifying the \ac{IDD} model to take wheel slip into account. 


Kinematic models
\citep{Anousaki2004}
\citep{Mandow2007}
\citep{Wang2015}
\citep{Baril2020}

Dynamic models
\citep{Yu2009}
\citep{Seegmiller2016}
\citep{Yang2022}

Quasi-dynamic models
\citep{Seegmiller2014}
\citep{Ostafew2016}
\citep{Rabiee2019}
\citep{Takemura2021}
Mention \citep{Seegmiller2013} for model identification.

Data-driven models
\citep{Williams2018}
\citep{Nagariya2020}
\citep{Tremblay2021}

\lightlipsum[1-3]

Datasets
\citep{Triest2022}


\subsection{Model training}

\begin{itemize}
	\item TODO : Discuss IPEM, Kalman Filters, datasets (Tartan, JFT, etc.) our method is applicable to any vehicle as long as...
\end{itemize}

\lightlipsum[1-3]
