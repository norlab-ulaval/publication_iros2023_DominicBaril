\section{Related Work}

\lightlipsum[1]

\subsection{Predictive motion models}

\begin{itemize}
	\item TODO : Discuss kinematic, quasi-dynamic, dynamic and data-driven
\end{itemize}

% Kinematic
The most commonly used motion models for \acp{UGV} are purely kinematic models.
Such models have the advantage that they are computationally simple and are relatively straightforward to train.
Since ideal models make the assumption that the \ac{UGV} operates in pure rolling and no slippage conditions, such models tend to have high motion prediction error.
For the case of \acp{SSMR}, which inherently require high wheel slippage and skidding to steer, \citet{Mandow2007} proposed the the \ICRBASED model, an extension of the \ac{IDD} model.
This model assumes that each robot wheel has a distinct \ac{ICR} due to slippage and that these remain constant for a given terrain. 
The authors have driven human-generated trajectories of about \SI{30}{\meter} and used \ac{RTK}-\ac{GNSS} to generate ground-truth to empirically fit model parameters.
\citet{Wang2015} later extended the \ICRBASED model, suggesting that wheel \ac{ICR} values based on the \ac{SSMR}'s \ac{ROC}. 
Alternatively, \citet{Anousaki2004} proposed to use empirical data to tune all coefficients of the Jacobian matrix, allowing to take wheel-terrain interactions into the motion model.
The original goal of this work is to provide a dead-reckoning motion prediction solution for \ac{SLAM} frameworks. 
In this work, the model was validated on a \ac{SSMR} on a human-driven trajectory of under \SI{50}{\meter}.
More recently, we have proposed an experimental evaluation of kinematic motion models for \acp{SSMR} operating on asphalt and snow terrains~\citep{Baril2020}. 
In this work, we show that kinematic models should be calibrated based on the desired prediction window and operating terrain. 
%% TODO: Find simple kinematic model for Ackerman vehicles?

% Dynamic
While accounting for wheel slip



Kinematic models
\citep{Anousaki2004}
\citep{Mandow2007}
\citep{Wang2015}
\citep{Baril2020}

Dynamic models
\citep{Yu2009}
\citep{Seegmiller2016}
\citep{Yang2022}

Quasi-dynamic models
\citep{Seegmiller2014}
\citep{Ostafew2016}
\citep{Rabiee2019}
\citep{Takemura2021}
Mention \citep{Seegmiller2013} for model identification.

Data-driven models
\citep{Williams2018}
\citep{Nagariya2020}
\citep{Tremblay2021}

\lightlipsum[1-3]

Datasets
\citep{Triest2022}


\subsection{Model training}

\begin{itemize}
	\item TODO : Discuss IPEM, Kalman Filters, datasets (Tartan, JFT, etc.) our method is applicable to any vehicle as long as...
\end{itemize}

\lightlipsum[1-3]
