%%%%%%%%%%%%%%%%%%%%%%%%%%%%%%%%%%%%%%%%%%%%%%%%%%%%%%%%%%%%%%%%%%%%%%%%%%%%%%%%
%2345678901234567890123456789012345678901234567890123456789012345678901234567890
%        1         2         3         4         5         6         7         8

\documentclass[letterpaper, 10 pt, conference]{ieeeconf}  % Comment this line out if you need a4paper
%\documentclass[a4paper, 10pt, conference]{ieeeconf}      % Use this line for a4 paper

\IEEEoverridecommandlockouts                              % This command is only needed if 
                                                          % you want to use the \thanks command
\overrideIEEEmargins                                      % Needed to meet printer requirements.

%--------------------------------------------
% Packages
\input{./latexGoodPractices/preamble}
\addbibresource{references.bib}
% \usepackage{float}
% \usepackage{graphics} % for pdf, bitmapped graphics files
% \usepackage{epsfig} % for postscript graphics files
% \usepackage{mathptmx} % assumes new font selection scheme installed
% \usepackage{times} % assumes new font selection scheme installed
% \usepackage{amsmath} % assumes amsmath package installed
% \usepackage{amssymb}  % assumes amsmath package installed

\title{\LARGE \bf 
    Doughnut Calib : A standardized framework to characterize mobile robot platform and evaluate predictive motion models
}


\author{Dominic Baril, Philippe Giguère, François Pomerleau
% \thanks{*This work was not supported by any organization}% <-this % stops a space
% \thanks{$^{1}$Albert Author is with Faculty of Electrical Engineering, Mathematics and Computer Science,
%         University of Twente, 7500 AE Enschede, The Netherlands
%         {\tt\small albert.author@papercept.net}}%
% \thanks{$^{2}$Bernard D. Researcheris with the Department of Electrical Engineering, Wright State University,
%         Dayton, OH 45435, USA
%         {\tt\small b.d.researcher@ieee.org}}%
}

% Only for reviews, remove for submission
\usepackage[switch]{lineno}
% Comment this line for submission
%\linenumbers

% acronym definitions
\acrodef{SLAM}{simultaneous localization and mapping}
\acrodef{SSMR}{skid-steering mobile robot}
\acrodef{AMR}{Ackermann mobile robot}
\acrodef{UGV}{unmanned ground vehicle}
\acrodef{IDD}{ideal differential-drive}
\acrodef{ICR}{instantaneous center or rotation}
\acrodef{RTK}{Realtime Kinematics}
\acrodef{GNSS}{Global Navigation Satellite System}
\acrodef{ROC}{radius of curvature}

% models
\newcommand{\ICRBASED}{\ac{ICR}-based\xspace} 

%-----------------------------

\begin{document}

\maketitle
\thispagestyle{empty}
\pagestyle{empty}

%%%%%%%%%%%%%%%%%%%%%%%%%%%%%%%%%%%%%%%%%%%%%%%%%%%%%%%%%%%%%%%%%%%%%%%%%%%%%%%%
\begin{abstract}

\lightlipsum[1-1]

\end{abstract}

%%%%%%%%%%%%%%%%%%%%%%%%%%%%%%%%%%%%%%%%%%%%%%%%%%%%%%%%%%%%%%%%%%%%%%%%%%%%%%%%

\section{Introduction}

\begin{itemize}
	\item TODO
\end{itemize}

\lightlipsum[1-1]

\begin{figure}[htbp]
	\centering
	\includegraphics[height=5cm]{example-image-a}
	\caption{TODO: Complete commands vs. effective input space vs. actual behaviour figure}
	\label{fig:robot_behaviour_intro}
\end{figure}

\lightlipsum[1-4]

%%%%%%%%%%%%%%%%%%%%%%%%%%%%%%%%%%%%%%%%%%%%%%%%%%%%%%%%%%%%%%%%%%%%%%%%%%%%%%%%

\section{Related Work}

\lightlipsum[1]

\subsection{Predictive motion models}

\begin{itemize}
	\item TODO : Discuss kinematic, quasi-dynamic, dynamic and data-driven
\end{itemize}

The computationally most simple models that have been proposed fall into the kinematic category.
Such models make use of a Jacobian matrix to compute the resulting robot body-frame velocities resulting from inputs.
\citet{Anousaki2004} were the first to suggest to use empirical data to tune all coefficients of the Jacobian matrix, allowing to take wheel-terrain interactions into the motion model.
The original goal of this work is to provide a dead-reckoning motion prediction solution for \ac{SLAM} frameworks. 
In this work, the model was validated on a \ac{SSMR} on a human-driven trajectory of under \SI{50}{\meter}.
\citet{Mandow2007} later proposed the \ICRBASED model, however this time by modifying the \ac{IDD} model to take wheel slip into account. 


Kinematic models
\citep{Anousaki2004}
\citep{Mandow2007}
\citep{Wang2015}
\citep{Baril2020}

Dynamic models
\citep{Yu2009}
\citep{Seegmiller2016}
\citep{Yang2022}

Quasi-dynamic models
\citep{Seegmiller2014}
\citep{Ostafew2016}
\citep{Rabiee2019}
\citep{Takemura2021}
Mention \citep{Seegmiller2013} for model identification.

Data-driven models
\citep{Williams2018}
\citep{Nagariya2020}
\citep{Tremblay2021}

\lightlipsum[1-3]

Datasets
\citep{Triest2022}


\subsection{Model training}

\begin{itemize}
	\item TODO : Discuss IPEM, Kalman Filters, datasets (Tartan, JFT, etc.) our method is applicable to any vehicle as long as...
\end{itemize}

\lightlipsum[1-3]


%%%%%%%%%%%%%%%%%%%%%%%%%%%%%%%%%%%%%%%%%%%%%%%%%%%%%%%%%%%%%%%%%%%%%%%%%%%%%%%%

\section{Methodology}

\begin{itemize}
	\item TODO (must) : Input-space characterization
	\item TODO (must) : Input-space random sampling
	\item TODO (must) : Model training
	\item TODO (nice to have) : Uncertainty characterization
\end{itemize}

\lightlipsum[1-4]

\begin{figure}[htbp]
	\centering
	\includegraphics[height=5cm]{example-image-b}
	\caption{Motion prediction pipeline (command -- > input vector --> body-frame velocity --> world frame pose)}
	\label{fig:b}
\end{figure}

\lightlipsum[1-12]

%%%%%%%%%%%%%%%%%%%%%%%%%%%%%%%%%%%%%%%%%%%%%%%%%%%%%%%%%%%%%%%%%%%%%%%%%%%%%%%%
\section{Results}

\lightlipsum[1]

\subsection{Experimental Setup}

\begin{itemize}
	\item TODO (must) : Description of all platforms
\end{itemize}

\lightlipsum[1]

\begin{figure}[htbp]
	\centering
	\includegraphics[height=4cm]{example-image-a}
	\caption{Various platforms used}
	\label{fig:test_platforms}
\end{figure}

\subsection{Improvement per calibration step}
\begin{itemize}
	\item TODO (must) : Show motion prediction improvement for every calibration step
\end{itemize}

\lightlipsum[1-3]

\begin{figure*}[htb]
	\centering
	\includegraphics[width=6.85in, height=4cm]{example-image-c}
	\caption{Large figure showing the improvement / calibration step for one or two most relevant models (kin / ML)}
	\label{fig:prediction_improvement_step}
\end{figure*}

\subsection{Benchmark of SOTA prediction models}

\begin{itemize}
	\item TODO (must) : Benchmark of model performance after completing the calibration
\end{itemize}

\lightlipsum[1-4]

\begin{figure*}[htp]
	\centering
	\includegraphics[width=6.85in, height=4cm]{example-image-b}
	\caption{Figure showing prediction performance per platform for all models after completing calibration steps}
	\label{fig:prediction_benchmark}
\end{figure*}








%%%%%%%%%%%%%%%%%%%%%%%%%%%%%%%%%%%%%%%%%%%%%%%%%%%%%%%%%%%%%%%%%%%%%%%%%%%%%%%%
\section{Conclusion}
\lightlipsum[1-2]


%%%%%%%%%%%%%%%%%%%%%%%%%%%%%%%%%%%%%%%%%%%%%%%%%%%%%%%%%%%%%%%%%%%%%%%%%%%%%%%%
\addtolength{\textheight}{-12cm}   % This command serves to balance the column lengths
                                  % on the last page of the document manually. It shortens
                                  % the textheight of the last page by a suitable amount.
                                  % This command does not take effect until the next page
                                  % so it should come on the page before the last. Make
                                  % sure that you do not shorten the textheight too much.

%%%%%%%%%%%%%%%%%%%%%%%%%%%%%%%%%%%%%%%%%%%%%%%%%%%%%%%%%%%%%%%%%%%%%%%%%%%%%%%%
\section*{ACKNOWLEDGMENT}

%%%%%%%%%%%%%%%%%%%%%%%%%%%%%%%%%%%%%%%%%%%%%%%%%%%%%%%%%%%%%%%%%%%%%%%%%%%%%%%%
\printbibliography

\end{document}


