%%%%%%%%%%%%%%%%%%%%%%%%%%%%%%%%%%%%%%%%%%%%%%%%%%%%%%%%%%%%%%%%%%%%%%%%%%%%%%%%
%2345678901234567890123456789012345678901234567890123456789012345678901234567890
%        1         2         3         4         5         6         7         8

\documentclass[letterpaper, 10 pt, conference]{ieeeconf}  % Comment this line out if you need a4paper
%\documentclass[a4paper, 10pt, conference]{ieeeconf}      % Use this line for a4 paper

\IEEEoverridecommandlockouts                              % This command is only needed if 
                                                          % you want to use the \thanks command
\overrideIEEEmargins                                      % Needed to meet printer requirements.

%--------------------------------------------
% Packages
\input{./latexGoodPractices/preamble}
\addbibresource{references.bib}
% \usepackage{float}
% \usepackage{graphics} % for pdf, bitmapped graphics files
% \usepackage{epsfig} % for postscript graphics files
% \usepackage{mathptmx} % assumes new font selection scheme installed
% \usepackage{times} % assumes new font selection scheme installed
% \usepackage{amsmath} % assumes amsmath package installed
% \usepackage{amssymb}  % assumes amsmath package installed

\title{\LARGE \bf 
    Object detection point clouds
}


\author{William Guimont-Martin, François Pomerleau, Philippe Giguère
% \thanks{*This work was not supported by any organization}% <-this % stops a space
% \thanks{$^{1}$Albert Author is with Faculty of Electrical Engineering, Mathematics and Computer Science,
%         University of Twente, 7500 AE Enschede, The Netherlands
%         {\tt\small albert.author@papercept.net}}%
% \thanks{$^{2}$Bernard D. Researcheris with the Department of Electrical Engineering, Wright State University,
%         Dayton, OH 45435, USA
%         {\tt\small b.d.researcher@ieee.org}}%
}

% Only for reviews, remove for submission
\usepackage[switch]{lineno}
% Comment this line for submission
%\linenumbers
%-----------------------------

\begin{document}

\maketitle
\thispagestyle{empty}
\pagestyle{empty}

%%%%%%%%%%%%%%%%%%%%%%%%%%%%%%%%%%%%%%%%%%%%%%%%%%%%%%%%%%%%%%%%%%%%%%%%%%%%%%%%
\begin{abstract}

\lightlipsum[1-1]

\end{abstract}

%%%%%%%%%%%%%%%%%%%%%%%%%%%%%%%%%%%%%%%%%%%%%%%%%%%%%%%%%%%%%%%%%%%%%%%%%%%%%%%%
\section{Introduction}

\begin{itemize}
    \item TODO
\end{itemize}

\lightlipsum[1-1]

\begin{figure}[htbp]
    \centering
    \includegraphics[height=5cm]{example-image-a}
    \caption{Cool figure}
    \label{fig:a}
\end{figure}

\lightlipsum[1-4]

%%%%%%%%%%%%%%%%%%%%%%%%%%%%%%%%%%%%%%%%%%%%%%%%%%%%%%%%%%%%%%%%%%%%%%%%%%%%%%%%
\section{Related Work}

\lightlipsum[1]

\subsection{Transformers in Computer Vision}

\begin{itemize}
    \item TODO
\end{itemize}

\lightlipsum[1-2]

\subsection{3D Object Detection}

\begin{itemize}
    \item TODO
\end{itemize}

\lightlipsum[1-3]

\subsection{3D Object Detection with Transformers}

\begin{itemize}
    \item TODO
\end{itemize}

\lightlipsum[1]

%%%%%%%%%%%%%%%%%%%%%%%%%%%%%%%%%%%%%%%%%%%%%%%%%%%%%%%%%%%%%%%%%%%%%%%%%%%%%%%%
\section{Method}

\lightlipsum[1-4]

\begin{figure}[htbp]
    \centering
    \includegraphics[height=5cm]{example-image-b}
    \caption{Figure explaining the network}
    \label{fig:b}
\end{figure}

\begin{figure*}[h]
    \centering
    \includegraphics[width=6.85in, height=4cm]{example-image-c}
    \caption{Complete architecture}
    \label{fig:c}
\end{figure*}

\lightlipsum[1-12]

%%%%%%%%%%%%%%%%%%%%%%%%%%%%%%%%%%%%%%%%%%%%%%%%%%%%%%%%%%%%%%%%%%%%%%%%%%%%%%%%
\section{Experiments}

\lightlipsum[1]

\subsection{Experimental Setup}

\lightlipsum[1]

\begin{figure}[htbp]
    \centering
    \includegraphics[height=4cm]{example-image-a}
    \caption{Qualitative results}
    \label{fig:e}
\end{figure}

\subsection{Comparison to State of the Art}

\begin{figure}[htbp]
    \centering
    \includegraphics[height=4cm]{example-image-b}
    \caption{Table comparing methods}
    \label{fig:e}
\end{figure}

\lightlipsum[1]

\subsection{Subsection}

\lightlipsum[1]

\subsection{Subsection}

\lightlipsum[1]

\subsection{Subsection}

\lightlipsum[1-2]

\begin{figure}[htbp]
    \centering
    \includegraphics[height=4cm]{example-image-b}
    \caption{Table comparing methods}
    \label{fig:e}
\end{figure}

\lightlipsum[1-2]

\begin{figure}[htbp]
    \centering
    \includegraphics[height=4cm]{example-image-b}
    \caption{Table comparing methods}
    \label{fig:e}
\end{figure}

\subsection{Subsection}

\lightlipsum[1-3]

\begin{figure*}[h]
    \centering
    \includegraphics[width=6.85in, height=4cm]{example-image-c}
    \caption{Table comparing methods}
    \label{fig:d}
\end{figure*}

\lightlipsum[1-2]

\lightlipsum[1-4]

%%%%%%%%%%%%%%%%%%%%%%%%%%%%%%%%%%%%%%%%%%%%%%%%%%%%%%%%%%%%%%%%%%%%%%%%%%%%%%%%
\section{Conclusion}
\lightlipsum[1]

%%%%%%%%%%%%%%%%%%%%%%%%%%%%%%%%%%%%%%%%%%%%%%%%%%%%%%%%%%%%%%%%%%%%%%%%%%%%%%%%
\addtolength{\textheight}{-12cm}   % This command serves to balance the column lengths
                                  % on the last page of the document manually. It shortens
                                  % the textheight of the last page by a suitable amount.
                                  % This command does not take effect until the next page
                                  % so it should come on the page before the last. Make
                                  % sure that you do not shorten the textheight too much.

%%%%%%%%%%%%%%%%%%%%%%%%%%%%%%%%%%%%%%%%%%%%%%%%%%%%%%%%%%%%%%%%%%%%%%%%%%%%%%%%
\section*{ACKNOWLEDGMENT}

%%%%%%%%%%%%%%%%%%%%%%%%%%%%%%%%%%%%%%%%%%%%%%%%%%%%%%%%%%%%%%%%%%%%%%%%%%%%%%%%
\printbibliography

\end{document}


