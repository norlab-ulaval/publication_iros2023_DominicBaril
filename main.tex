%%%%%%%%%%%%%%%%%%%%%%%%%%%%%%%%%%%%%%%%%%%%%%%%%%%%%%%%%%%%%%%%%%%%%%%%%%%%%%%%
%2345678901234567890123456789012345678901234567890123456789012345678901234567890
%        1         2         3         4         5         6         7         8

\documentclass[letterpaper, 10 pt, conference]{ieeeconf}  % Comment this line out if you need a4paper


\usepackage{tikz}
\newcommand*\circled[1]{\tikz[baseline=(char.base)]{
		\node[shape=circle,draw,inner sep=1pt] (char) {#1};}}

\input{latexGoodPractices/preamble.tex}


%%%%%%%%%%%%%%%%%%%%%%%%%%%%%%%%%

\setlength{\parskip}{0pt}

\addbibresource{references.bib}

% Philippe's nuclear buttons to squeeze more text in the paper, use with care!
%\linepenalty=3000  % Works well
%\addtolength{\abovecaptionskip }{-0.02in}% Works well
%\addtolength{\belowcaptionskip }{-0.05in}% DB and SP homebrewed nuclear button
%\addtolength{\textfloatsep}{-0.1in}
%\addtolength{\dblfloatsep }{-0.1in}

%\documentclass[a4paper, 10pt, conference]{ieeeconf}      % Use this line for a4 paper

\IEEEoverridecommandlockouts                              % This command is only needed if 
% you want to use the \thanks command

\overrideIEEEmargins                                      % Needed to meet printer requirements.

%In case you encounter the following error:
%Error 1010 The PDF file may be corrupt (unable to open PDF file) OR
%Error 1000 An error occurred while parsing a contents stream. Unable to analyze the PDF file.
%This is a known problem with pdfLaTeX conversion filter. The file cannot be opened with acrobat reader
%Please use one of the alternatives below to circumvent this error by uncommenting one or the other
%\pdfobjcompresslevel=0
%\pdfminorversion=4

% See the \addtolength command later in the file to balance the column lengths
% on the last page of the document

\title{\LARGE \bf
	% Instance segmentation of wood logs for automation of log grasping in forestry operations
	Instance Segmentation for Autonomous Log Grasping in Forestry Operations
}
\author{Jean-Michel Fortin$^{1}$, Olivier Gamache$^{1}$, Vincent Grondin$^{1}$, François Pomerleau$^{1}$, Philippe Gigu�?re$^{1}$ %\\
	%Northern Robotics Laboratory, Université Laval, Québec, Québec, Canada \\
	\thanks{$^{1}$ The authors are with Northern Robotics Laboratory, Université Laval, Québec City, Canada,
		{\tt{\small{$\{$jean-michel.fortin, olivier.gamache, vincent.grondin, francois.pomerleau$\}$ @norlab.ulaval.ca,} philippe.giguere@ift.ulaval.ca}}}%
	\thanks{* This research was supported by the Natural Sciences and Engineering Research Council of Canada (NSERC) through the grant CRD 538321-18, in collaboration with FP Innovations and Resolute Forest Products.}%
	\thanks{** The code and dataset for this paper are available here : \url{https://github.com/norlab-ulaval/logpiles_segmentation}}%
}

\begin{document}
	
	\maketitle
	\thispagestyle{empty}
	\pagestyle{empty}
	
	%%%%%%%%%%%%%%%%%%%%%%%%%%%%%%%%%%%%%%%%%%%%%%%%%%%%%%%%%%%%%%%%%%%%%%%%%%%%%%%%
	\begin{abstract}
		
		
		
	\end{abstract}
	
	%%%%%%%%%%%%%%%%%%%%%%%%%%%%%%%%%%%%%%%%%%%%%%%%%%%%%%%%%%%%%%%%%%%%%%%%%%%%%%%%
	\section{INTRODUCTION}
	
	
	%%%%%%%%%%%%%%%%%%%%%%%%%%%%%%%%%%%%%%%%%%%%%%%%%%%%%%%%%%%%%%%%%%%%%%%%%%%%%%%%
	\section{RELATED WORK}
	
	
	
	
	%%%%%%%%%%%%%%%%%%%%%%%%%%%%%%%%%%%%%%%%%%%%%%%%%%%%%%%%%%%%%%%%%%%%%%%%%%%%%%%%
	\section{EXPERIMENTS \& RESULTS}
	
	
		%%%%%%%%%%%%%%%%%%%%%%%%%%%%%%%%%%%%%%%%%%%%%%%%%%%%%%%%%%%%%%%%%%%%%%%%%%%%%%%%
		\section{CONCLUSION}
		

		
		% \addtolength{\textheight}{-0cm}   % (Original: 12cm)This command serves to balance the column lengths
		%                                   % on the last page of the document manually. It shortens
		%                                   % the textheight of the last page by a suitable amount.
		%                                   % This command does not take effect until the next page
		%                                   % so it should come on the page before the last. Make
		%                                   % sure that you do not shorten the textheight too much.
		
		%%%%%%%%%%%%%%%%%%%%%%%%%%%%%%%%%%%%%%%%%%%%%%%%%%%%%%%%%%%%%%%%%%%%%%%%%%%%%%%%
		%\section*{APPENDIX}
		
		%Appendixes should appear before the acknowledgment.
		
		%\section*{ACKNOWLEDGMENT}
		
		
		
		% This research was supported by the  Natural  Sciences and Engineering  Research  Council of  Canada  (NSERC)  through the grant CRDPJ 527642-18 SNOW (Self-driving Navigation Optimized for Winter). 
		% Thanks to Resolute Forest Products, FPInnovations and Domtar Windsor for the help with the data collection.
		% Thanks to the FORAC consortium.
		
		%The preferred spelling of the word acknowledgment in America is without an e after the g. Avoid the stilted expression, One of us (R. B. G.) thanks . . .  Instead, try R. B. G. thanks. Put sponsor acknowledgments in the unnumbered footnote on the first page.
		
		
		\printbibliography
		
	\end{document}